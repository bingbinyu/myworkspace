\documentclass{ti2}

\usepackage{listings}

% Dateikodierung ist utf8
\usepackage[utf8]{inputenc}   

\begin{document}

%damit der Code nicht abgeschnitten wird, falls er zu lang ist.
\lstset{linewidth=\linewidth,breaklines=true}
% Nr, Abgabedatum, Gruppenleiter, Gruppenname, Name1...Name4
\Abgabeblatt{3}{21.11.2016}{Marc Hildebrandt}{C08}%
                {Timo Jasper (Inf, 3.FS.)}{Thomas Tannous (Inf, 3.FS.)}%
                {Oliver Hilbrecht (Inf, 3.FS.)}{Moritz Gerken (Inf, 3.FS.)}%

\section*{Aufgabe 1}

Der Prozess `ls -lR / $>$ausgabe`, welcher im foreground im User-Mode läuft, wird durch das Drücken von Strg+z
aufgehalten, indem das Signal SIGSTOP gesendet wird. Im gleichen Moment wird durch das drücken der Tasten, der aktuelle Prozess unterbrochen (interrupted), da der Computer
den input der Tastatur verarbeiten muss. 
ls listet aus dem root Verzeichnis rekursiv alle Dateiein und ruft somit read auf, was einen Systemaufruf darstellt und den Prozess in die Kernel-Routine bringt (Trap).
Durch die Rekursion werden Verzeichnisse aufgerufen, so kommt es auch zum Systemaufruf der funktion open bzw. readdir (trap). 
Falls die Datei ausgabe noch nicht erstellt wird sie erstellt durch den Systemaufruf open (trap). In die Datei wird geschrieben mit dem Systemaufruf write (trap).



\section*{Aufgabe 2}

Die Aufgabe war es einen Handler für das Signal USR1 zu erstellen. 
Die Funktion \emph{handle\_signal()} wird aufgerufen, sobald
ein USR1 Signal empfangen wird. Diese kriegt dann Informationen über den Sender des 
Signals über \emph{siginfo\_t}.

\lstinputlisting[language=C++]{./src/sigserver.c}


%TODO ....................................................

\section*{Aufgabe 3}

\subsection*{3.a)}
%Zu welchen unangenehmen Effekten würde dieser Scheduling-Algorithmus führen, wenn
%der CPU-Kontostand der Prozesse nicht nach jeder Zeitscheibe halbiert würde?

Große Prozesse mit langer Laufzeit würden ihren Kontostand trotz möglicherweise Hoher Priorität schnell erhöhen und dann 'verhungern', da ihr Kontostand sie daran hindern würde einen niedrigen Wert bei der Effektiven Priorität zu erlangen. Prozesse mit kurzer laufzeit würden davon profitieren, da sie nur wenige Zeitscheiben benötigen um abgearbeitet zu sein.

\subsection*{3.b)}
%A und B sind immer lauffähige Prozesse. A hat die Basispriorität 56, B dagegen die
%Basispriorität 10. Wann laufen diese Prozesse? Berechnet eine Tabelle für die ersten zwölf
%Zeitscheiben der Laufzeit. Lässt das Ergebnis eine Regelmäßigkeit erkennen? Wenn ja:
%welche?

\begin{tabular}{l|c c c c c c c c c c c c}
	Prozess A (nice = 56)            & & & & & & & & & & & & \\
	\hline
	Voriges CPU-Konto                &   & 0 & 0  &  50&  25&  63&  32&  16&  58&  29&  15&  58\\
	$\Downarrow$ + Aufschlag (\%CPU) &   & 0 & 100&  50& 125&  63&  32& 116&  58&  29& 115&  58\\
	$\Downarrow$ Veraltern (/2)      &   &   &    &    &    &    &    &    &    &    &    &    \\
	Aktuelles CPU-Konto              &   & 0 &  50&  25&  63&  32&  16&  58&  29&  15&  58&  29\\
	$\Downarrow$ + Basispriorität    &   &   &    &    &    &    &    &    &    &    &    &    \\
	Effektive Priorität              & 56& 56& 105&  81& 119&  88&  72& 114&  85&  71& 113&  85\\
	\hline
	\hline
	Prozess B (nice = 10)            & & & & & & & & & & & & \\
	\hline
	Voriges CPU-Konto                &   & 0  & 50&  25&  63&  32&  66&  83&  42&  71&  86&  43\\
	$\Downarrow$ + Aufschlag (\%CPU) &   & 100& 50& 125&  63& 132& 166&  83& 142& 171&  86& 143\\
	$\Downarrow$ Veraltern (/2)      &   &    &   &    &    &    &    &    &    &    &    &    \\
	Aktuelles CPU-Konto              &   & 50 & 25&  63&  32&  66&  83&  42&  71&  86&  43&  72\\
	$\Downarrow$ + Basispriorität    &   &    &   &    &    &    &    &    &    &    &    &    \\
	Effektive Priorität              & 10& 60 & 35&  98&  42&  76&  93&  52&  81&  96&  53&  82\\
\end{tabular}\\

Nach der einfädelung in den ersten 2 Zeitscheiben, sieht es so aus, als würde Prozess B doppelt so oft den Prozessor zugesprochen bekommen, wie Prozess A.\\

Es wurde auf ganze zahlen gerundet, da wir ansonsten unnötig große Nachkommazahlen erhalten würden.

\subsection*{3.c)}
%A und B sind immer lauffähige Prozesse. A hat die Basispriorität 0. Ab welcher Basispriorität
%würde B die CPU niemals erhalten?

Bei dem Algorithmus dieser Aufgabenstellung ist in der unten stehenden Tabelle leicht erkennbar, dass sich bei Prozess A (ggf. er läuft ununterbrochen) die 100 als CPU-Konto-Wert einpendelt. Wenn Prozess B die Basisprioriät 100 zugeteilt würde, dann kähme er von Beginn an niemals an eine Zeitscheibe heran, da bei gleicher Effektiven Priorität aufgrund seines Namens (Alphabetische Ordnung) Prozess A immernoch bevorzugt werden würde. Bei 99 und weniger Prioritäts-Wert hätte Prozess B den Prozessor wenigstens einmal im 8. Durchlauf (Prozess A stand zum erstenmal auf 100) erhalten. Eine geringere Basispriorität als die von 100 wird nach Konvention nicht vergeben.\\


\begin{tabular}{l|c c c c c c c c c c c c c}
	Prozess A (nice = 0)             & & & & & & & & & & & & \\
	\hline
	Voriges CPU-Konto                &  &   0&  50&  75&  88&  94&  97&  99& 100& 100& 100& 100& ...\\
	$\Downarrow$ + Aufschlag (\%CPU) &  & 100& 150& 175& 188& 194& 197& 199& 200& 200& 200& 200& ...\\
	$\Downarrow$ Veraltern (/2)      &  &    &    &    &    &    &    &    &    &    &    &    &    \\
	Aktuelles CPU-Konto              &  &  50&  75&  88&  94&  97&  99& 100& 100& 100& 100& 100& ...\\
	$\Downarrow$ + Basispriorität    &  &    &    &    &    &    &    &    &    &    &    &    &    \\
	Effektive Priorität              & 0&  50&  75&  88&  94&  97&  99& 100& 100& 100& 100& 100& ...\\
\end{tabular}

Es wurde auf ganze zahlen gerundet, da wir ansonsten unnötig große Nachkommazahlen erhalten würden.

\subsection*{3.d)}

\begin{tabular}{l|c c c c c c c c c c c c}
	Prozess A (nice = 0)            & & & & & & & & & & & & \\
	\hline
	Voriges CPU-Konto                &   &   0&  50&  61&  81&  77&  89&  81&  91&  82&  91&  82\\
	$\Downarrow$ + Aufschlag (\%CPU) &   & 100& 122& 161& 153& 177& 161& 181& 163& 182& 163& 182\\
	$\Downarrow$ Veraltern (/2)      &&&&&&&&&&&&\\
	Aktuelles CPU-Konto              &   &  50&  61&  81&  77&  89&  81&  91&  82&  91&  82&  91\\
	$\Downarrow$ + Basispriorität    &&&&&&&&&&&&\\
	Effektive Priorität              &  0&  50&  61&  81&  77&  89&  81&  91&  82&  91&  82&  91\\
	\hline
	\hline
	Prozess B (nice = 15)            & & & & & & & & & & & & \\
	\hline
	Voriges CPU-Konto                &   &   0&   0&  14&   7&  33&  17&  23&  12&  20&  10&  19\\
	$\Downarrow$ + Aufschlag (\%CPU) &   &   0&  28&  14&  35&  33&  45&  23&  40&  20&  38&  19\\
	$\Downarrow$ Veraltern (/2)      &&&&&&&&&&&&\\
	Aktuelles CPU-Konto              &   &   0&  14&   7&  18&  17&  23&  12&  20&  10&  19&  10\\
	$\Downarrow$ + Basispriorität    &&&&&&&&&&&&\\
	Effektive Priorität              & 15&  15&  29&  22&  33&  32&  38&  27&  35&  25&  34&  38\\
\end{tabular}

Wenn Prozess B die Zeitscheibe erhält, arbeitet er 55 ms und gibt die Zeitscheibe ab (wir gehen in dieser Rechung davon aus, dass die neue Verteilung 0 ms in Anspruch nimmt), daher verliert Prozess B bereits in der selben Zeitscheibe 145 ms seiner Sperrzeit von 250 ms Vorrausgesetzt Prozess B hat nur mit Prozessen zu tun, die ihre Zeitscheiben voll ausnutzen ist er effektiv für eine weitere Zeitscheibe blockiert, wenn er write() aufgerufen hat. Prozess B nimmt pro erhaltener Zeittafel 28\% von dieser ein, die restlichen 72\% bekommt der nächst Priorisierte Prozess zugesprochen in diesem Fall immer Prozess A.\\

Es wurde auf ganze zahlen gerundet, da wir ansonsten unnötig große Nachkommazahlen erhalten würden.

\section*{Weitere Aufgaben}



\end{document}
