% !TeX root = ../02_C01_habermann_koester_rohde.tex

\subsection{A)}
    Man erkennt, dass hier jeweils immer nur abwechselnd Sheduled wird.
    Dadurch, dass nicht geteilt wird, werden Kontostände einfach immer nur um 100 „Punkte“ erhöht.
    Dadruch ist es natürlich logisch, dass die Prozesse immer abwechselnd laufen.
    Dies variiert auch keineswegs.

    \begin{center}
        \begin{tabular}{|c|c|c|c|c|c|c|} \hline
            Konto A     &   akt. Wert        &   0   &   0       &   0       &   100     &   100     \\  \hline
            Nutzung A   &   100o0           &   0   &   0       &   100     &   0       &   100     \\  \hline
            Konto A'    &   trunc(konto+nutz/2)     &   0   &   0       &   100     &   100     &   200     \\  \hline
            Prio A      &   basisprio+konto'     &   56  &   56      &   156     &   156     &   256     \\  \hline\hline
            Konto B     &   akt Wert        &   0   &   0       &   100     &   100     &   200     \\  \hline
            Nutzung     &   100o0          &   0   &   100     &   0       &   100     &   0       \\  \hline
            Konto B'    &   trunc(konto+nutz/2)      &   0   &   100     &   100     &   200     &   200     \\  \hline
            Prio B      &   bprio+kont'     &   10  &   110     &   110     &   210     &   210     \\  \hline
        \end{tabular}
    \end{center}

\subsection{B)}
    \begin{center}
        \resizebox{\textwidth}{!}{%
            \begin{tabular}{|c|c|c|c|c|c|c|c|c|c|c|c|c|c|c|c|}  \hline
                Konto A     &           &   0   &   0       &   0       &   50      &   25      &   12      &   56      &   28      &   14      &   57      &   28      &   14      &   57      &   \ldots  \\  \hline
                Nutzung     &             &   0   &   0       &   100     &   0       &   0       &   100     &   0       &   0       &   100     &   0       &   0       &   100     &   0       &   \ldots  \\  \hline
                Konto A'    &        &   0   &   0       &   50      &   25      &   12      &   56      &   28      &   14      &   57      &   28      &   14      &   57      &   28      &   \ldots  \\  \hline
                Prio A      &        &   56  &   56      &   106     &   81      &   68      &   112     &   84      &   70      &   113     &   84      &   70      &   113     &   84      &   \ldots  \\  \hline
                \multicolumn{16}{|c|}{ab hier: wiederkehrendes Muster(3f. auf 10ter Takt)}                                                                                                                                              \\  \hline
                Konto B     &           &   0   &   0       &   50      &   25      &   62      &   81      &   40      &   70      &   85      &   42      &   71      &   85      &   42      &   \ldots  \\  \hline
                Nutzung B   &             &   0   &   100     &   0       &   100     &   100     &   0       &   100     &   100     &   0       &   100     &   100     &   0       &   100     &   \ldots  \\  \hline
                Konto B'    &        &   0   &   50      &   25      &   62      &   81      &   40      &   70      &   85      &   42      &   71      &   85      &   42      &   71      &   \ldots  \\  \hline
                Prio B      &        &   10  &   60      &   35      &   72      &   91      &   50      &   80      &   95      &   52      &   82      &   95      &   52      &   82      &   \ldots  \\  \hline
            \end{tabular}
        } % ende der resizebox
    \end{center}

\subsection{C)}
    \begin{center}
        \begin{tabular}{|c|c|c|c|c|c|c|c|c|c|c|c|}  \hline
            Konto A     &   0   &   0       &   50      &   75      &   87      &93     &   96      &   98      &   99      &   99  \\  \hline
            Nutzung A   &   0   &   100     &   100     &   100     &   100     &100    &   100     &   100     &   100     &   199 \\  \hline
            Konto A'    &   0   &   50      &   75      &   87      &   93      &96     &   98      &   99      &   99      &   99  \\  \hline
            Prio A      &   0   &   50      &   75      &   87      &   93      &96     &   98      &   99      &   99      &   99  \\  \hline\hline
            Konto B     &   0   &   0       &   0       &   0       &   0       &0      &   0       &   0       &   0       &   0   \\  \hline
            Nutzung B   &   0   &   0       &   0       &   0       &   0       &0      &   0       &   0       &   0       &   0   \\  \hline
            Konto B'    &   0   &   0       &   0       &   0       &   0       &0      &   0       &   0       &   0       &   0   \\  \hline
            Prio B      &   99  &   99      &   99      &   99      &   99      &99     &   99      &   99      &   99      &   99  \\  \hline
        \end{tabular}
    \end{center} \par\medskip

    Wir kommen in eine Endlosschleife wo immer A gewählt wird, da wir immer bei 99,5 die 5 wegschneiden.
    Wir wählen A aus, da wir alphabetisch vorgehen(Tutorium).
    Am Ende sieht man, dass bei 99 immer wieder alles gleich aussieht es wird sich also in der Matrix nichts verändern.
    Desweiteren wird nun 98 untersucht, Hier nehmen wir an, dass wir ein Wechsel bekommen.
    Dadurch ist bewiesen, dass es keine niendiger Zahl mehr gibt, wo B komplett ignoriert wird. \par

    Prio99: \par\medskip

    \begin{center}
        \begin{tabular}{|c|c|c|c|c|c|c|c|c|c|c|}    \hline
            Konto A     &   0   &0      &   50      &   75      &   87      &   93      &   96      &   98      &   99      \\  \hline
            Nutzung A   &   0   &100    &   100     &   100     &   100     &   100     &   100     &   100     &   0       \\  \hline
            Konto A'    &   0   &50     &   75      &   87      &   93      &   96      &   98      &   99      &   49      \\  \hline
            Prio A      &   0   &50     &   75      &   87      &   93      &   96      &   98      &   99      &   49      \\  \hline\hline
            Konto B     &   0   &0      &   0       &   0       &   0       &   0       &   0       &   0       &   0       \\  \hline
            Nutzung B   &   0   &0      &   0       &   0       &   0       &   0       &   0       &   0       &   100     \\  \hline
            Konto B'    &   0   &0      &   0       &   0       &   0       &   0       &   0       &   0       &   50      \\  \hline
            Prio B      &   98  &98     &   98      &   98      &   98      &   98      &   98      &   98      &   98      \\  \hline
        \end{tabular}
    \end{center} \par\medskip

    Man erkennt, dass hier der Scheduler einen Lauf an b abgibt.
    Deswegen stimmt es dass die Scheduling Zahl(nice-Wert[BasisPrio]) immer $<$99 sein muss in diesem Algorithmus.
    
    
\subsection{D)}
<<<<<<< HEAD
Da wir hier nur 55 ms des Takts laufen, liegt die Nutzung bei $\frac{55}{255}$ also 0,22 was 22\% ergibt.
=======
Da wir hier nur 55 ms des Takts laufen, liegt die Nutzung bei $\frac{55}{250}$ also 0,22 was 2\% ergibt.
>>>>>>> 8c33660a3c48571d9205854189a0f6ef720c82a9
    \begin{center}
        \resizebox{\textwidth}{!}{%
            \begin{tabular}{|c|c|c|c|c|c|c|c|c|c|c|c|c|c|c|c|}  \hline  
            Konto A     &               &   0   &   0       &   57       &   67      &   83      &  80       &   90      &   84     &   92      &   85      &   92      &   85      &  92      &   \ldots  \\  \hline
                Nutzung     &            &      &   100       &   78      &   100     &   78      &   100     &   78      &   100    &   78     &   100       &   78       &   100     &   78       &   \ldots  \\  \hline
                Konto A'    &       &   0   &   57       &   67      &   83      &   80      &   90      &   84      &   92     &   85      &   92      &   85   &   92      &   85     &   \ldots  \\  \hline
                Prio A      &        &   15  &   72      &   82       &   98      &   95      &   105     &   99      &   107    &   100     &   107      &   100      &   107     &   100      &   \ldots  \\  \hline\hline

                Konto B     &          &   0   &   0       &   0       &   11       &   5       &   13      &   6      &   14      &  7     &   14      &   7       &  14       &   7      &   \ldots  \\  \hline
                Nutzung B   &             &   0   &   0       &   22      &  0         &   22      &   0       &   22     &   0     &   22       &   0     &   22      &   0       &   22     &   \ldots  \\  \hline
                Konto B'    &        &   0   &   0       &   11      &   5        &   13      &   6       &   14     &   7      &   14      &   7      &   14      &   7       &   14      &   \ldots  \\  \hline
                Prio B      &        &   0  &    0       &   11      &   5        &   13      &   6       &   14     &   7      &   14      &   7      &   14      &   7       &   14      &   \ldots  \\  \hline 
            \end{tabular}
        } % ende der resizebox
    \end{center}
