% !TeX root = ../02_C01_habermann_koester_rohde.tex

\subsection{Signalhandler}
    \codeSh{handleSignal}{sigserver.cc}{src/sigserver.cc}

    Wir haben einen Signalhandler gebaut, der als Parameter die drei Rueckgaben vom \C{sigaction} bekommt und daraus eine triviale Konsolenausgabe gemaess Aufgabenstellung baut.

\subsection{main}
    \codeSh{main}{sigserver.cc}{src/sigserver.cc}

    Unsere \m{main}-Methode liefert zuerst die eigene PID zurueck, damit wir direkt den \C{kill}-Befehl darauf absetzen koennen.
    Dann initialisieren wir unser \C{sigaction}, setzen die Flags und den Namen der Methode, die ein Signal verarbeiten soll und registrieren zuletzt den "Listener" fuer das Signal \m{SIGUSR1}.
    Die Methode \m{pause()} laesst das Programm so lange warten, bis ein Signal gesendet wird. Danach beendet es sich mit dem Exit-Code 0.
