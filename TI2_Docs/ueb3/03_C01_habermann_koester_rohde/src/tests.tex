% !TeX root = ../03_C01_habermann_koester_rohde.tex

\subsubsection{Standard-Aufruf}
    Wir starten unser Programm und lassen es mittels \C{bg} im Hintergrund laufen. Dann senden wir das Signal SIGUSR1 an den Prozess:

    \begin{lstlisting}[numbers=none]
x13->./sigserver
My PID is: 8767
^Z
[1]+  Stopped       ./sigserver

x13->bg
[1]+ ./sigserver &

x13->kill -SIGUSR1 8767
Received signal #10 from PID 24655, running on UID 14104
    \end{lstlisting}

\subsubsection{Aufruf mit relativer Prozess-ID}
    Wir starten unser Programm und lassen es mittels \C{bg} im Hintergrund laufen. Dann senden wir das Signal SIGUSR1 an die relative Prozess-ID:

    \begin{lstlisting}[numbers=none]
x13->./sigserver
My PID is: 9597
^Z
[1]+  Stopped       ./sigserver

x13->bg
[1]+ ./sigserver &

x13->kill -SIGUSR1 %1
Received signal #10 from PID 24655, running on UID 14104
    \end{lstlisting}

\subsubsection{Direkter Aufruf im Hintergrund}
    Wir starten unser Programm und lassen es mittels \C{ \&} im Hintergrund laufen. Dann senden wir das Signal SIGUSR1 an die Prozess-ID:

    \begin{lstlisting}[numbers=none]
x13->./sigserver &
[1] 9764
x13->My PID is: 9764
kill -SIGUSR1 9764
Received signal #10 from PID 24655, running on UID 14104
    \end{lstlisting}

\subsubsection{Verwendung von Signal-ID 10}
    Da \C{SIGUSR1} in der Dezimaldarstellung eine 10 ist, ist es eigendlich unnoetig - dennoch haben wir alle Tests mit dem Aufruf \C{kill -10 (PID)} wiederholt. \\
    Da die Ergebnisse absolut gleich sind, verzichten wir hier auf eine doppelte Dokumentation.
