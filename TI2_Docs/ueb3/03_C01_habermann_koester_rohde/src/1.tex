% !TeX root = ../02_C01_habermann_koester_rohde.tex

\begin{itemize}
    
    \item Bei der Eingabe des Befehls \m{ls -lR} in die Shell loest diese einen Trap aus (Systemaufruf) 
    \begin{itemize}
       \item Zusätzlich wird die Bash mit den Signal  SIGSTP (20) gestoppt
\end{itemize}  
    \item Der Befehl CTRL-Z loest einen Interrupt aus und
        \begin{itemize}
            \item sendet das Signal SIGSTP (20) an das Programm, um es zu pausieren
        \end{itemize}
    \item Der Befehl \m{bg} sendet SIGCONT (18), um das Programm fortzusetzen
    \item Falls die maximale Dateigroesse ueberschritten wird, wird ein SIGXFSZ ausgelöst
    
    \item SIGTTOU (22) wird aufgerufen, um weiter im Hintergrund mittels der Shell auf der Datei \" Ausgabe\" zu schreiben.
    \item Nachdem die Datei fertig beschrieben wurde wird der Prozess im Hintergrund mittels dem Singnal SIGSTOP(19) gestoppt.
    
\end{itemize}
