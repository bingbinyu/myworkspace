% !TeX root = ../05_C01_habermann_koester_rohde.tex


\begin{lstlisting}[numbers=none]
int f = open(``/home/ti2/archive/nikolaus.avi'', O_RDONLY);
\end{lstlisting}

\m{open} stellt die Verbindung zwischen der Datei \C{nikolaus.avi} und dem ``file descriptor'' \C{f} her. \\
Es ist also nach unserem Verständnis quasi ein Pointer auf die Datei.\\ Deshalb wird die Datei geöffnet, aber nicht geladen.

\begin{itemize}
\item Die Inode 0 (aus Block 2);
\item der Datenblock der Verzeichnisdatei ``/'', dessen Nummer in Inode 0 steht;
\item Inode 36 aus Datenblock 9;
\item Datenblock der Verzeichnisdatei ``home'', extrahiert aus Inode 36;
\item Inode 99 aus Datenblock 2000;
\item Datenblock der Verzeichnisdatei ``ti2'', extrahiert aus Inode 99;
\item Inote 206 aus Datenblock 3101;
\item Datenblock der Verzeichnisdatei ``archive'', extrahiert aus Inode 206;
\item Inode 12783 aus Datenblock 50.
%\item 24.576 Datenblöcke der Datei ``nikolaus.avi'', extrahiert aus Inode 12783.
\end{itemize}


\begin{lstlisting}[numbers=none]
int f = open(``/home/ti2/meta'', O_RDWR);
\end{lstlisting}

Es wird geladen:

\begin{itemize}
\item Die Inode 0 (aus Block 2);
\item der Datenblock der Verzeichnisdatei ``/'', dessen Nummer in Inode 0 steht;
\item Inode 36 aus Datenblock 9;
\item Datenblock der Verzeichnisdatei ``home'', extrahiert aus Inode 36;
\item Inode 99 aus Datenblock 2000.
%\item Datenblock der Datei ``meta'', extrahiert aus Inode 99.

\end{itemize}


\begin{lstlisting}[numbers=none]
lseek(f,-10000, SEEK_END);
\end{lstlisting}

\C{lseek} setzt den Pointer in \C{f} 10.000 Zeichen vor das Dateiende.

\begin{lstlisting}[numbers=none]
count = read(f, buf, 4096);
\end{lstlisting}
\C{count} liest 4096 B ab der, durch \C{lseek} gesetzten, Position und speichert diese in \m{buf}.\\ Dies sind die letzten $9$ Blöcke ($\frac{4096}{508}=8.06$, da 4B je Block durch die Blocknummer belegt sind) der Datei ``nikolaus.avi'', deren Adressen wir aus (einem der Indirektblöcke) der Inode 12783 (im Datenblock 50) auslesen müssen.\\

\begin{lstlisting}[numbers=none]
write(g, buf, count);
\end{lstlisting}

Schließlich schreiben wir die gelesenen 4096 B aus \m{buf} in \m{g} (also in die Datei \m{meta}). \\
Die Inode 99 aus Datenblock 2000 enthält die Informationen über den Datenblock der Datei ``meta'', welcher leer ist (abgesehen von der Datenblocknummer, die die ersten 4B belegt).\\
Wir schreiben hier in 9 Blöcke, die in den Direktblock der o.a. Inode passen.