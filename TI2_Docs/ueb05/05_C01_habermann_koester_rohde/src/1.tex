% !TeX root = ../05_C01_habermann_koester_rohde.tex

\subsection{virtueller Adressraum des Prozesses, den die Funktionen einnehmen}

Um den Adressraum zu extrahieren, haben wir im gegebenen Programm mittels \C{gdb} breakpoints jeweils an den Anfang und das Ende der beiden Funktionen gesetzt.\\
Ergebnis:\\

Anfang \m{main}: 0x000000000040057d
\begin{lstlisting}[numbers=none]
Breakpoint 1, 0x000000000040057d in main ()

(gdb) info frame
Stack level 0, frame at 0x7fffffffe6f0:
 rip = 0x40057d in main; saved rip 0x7ffff7a6eead
 Arglist at 0x7fffffffe6e0, args:
 Locals at 0x7fffffffe6e0, Previous frame's sp is 0x7fffffffe6f0
 Saved registers:
  rip at 0x7fffffffe6e8
(gdb) info stack
#0  0x000000000040057d in main ()
\end{lstlisting}

Ende \m{main}: 0x00000000004005e5
\begin{lstlisting}[numbers=none]
Breakpoint 4, 0x00000000004005e5 in main ()
(gdb) info frame
Stack level 0, frame at 0x7fffffffe6e8:
 rip = 0x4005e5 in main; saved rip 0x7ffff7a6eead
 Arglist at unknown address.
 Locals at unknown address, Previous frame's sp is 0x7fffffffe6f0
 Saved registers:
  rip at 0x7fffffffe6e8

\end{lstlisting}

\clearpage

Anfang \m{factorial}: 0x000000000040054c
\begin{lstlisting}[numbers=none]
Breakpoint 3, 0x000000000040054c in factorial ()
(gdb) info frame
Stack level 0, frame at 0x7fffffffe6c0:
 rip = 0x40054c in factorial; saved rip 0x4005c6
 called by frame at 0x7fffffffe6f0
 Arglist at 0x7fffffffe6b0, args:
 Locals at 0x7fffffffe6b0, Previous frame's sp is 0x7fffffffe6c0
 Saved registers:
  rip at 0x7fffffffe6b8
(gdb) info stack
#0  0x000000000040054c in factorial ()
#1  0x00000000004005c6 in main ()
\end{lstlisting}

Ende \m{factorial}: 0x000000000040057c
\begin{lstlisting}[numbers=none]
Breakpoint 4, 0x000000000040057c in factorial ()
(gdb) info frame
Stack level 0, frame at 0x7fffffffe6b8:
 rip = 0x40057c in factorial; saved rip 0x4005c6
 called by frame at 0x7fffffffe6f0
 Arglist at 0x7fffffffe6e0, args:
 Locals at 0x7fffffffe6e0, Previous frame's sp is 0x7fffffffe6c0
 Saved registers:
  rip at 0x7fffffffe6b8
(gdb) info stack
#0  0x000000000040057c in factorial ()
#1  0x00000000004005c6 in main ()
\end{lstlisting}

Auch zu sehen sind diese Adressen, wenn man die Funktionen main und factorial dissasambelt.\\
Eine Ausdehnung findet in strtol und printf (dazu später mehr) statt.
\begin{lstlisting}[numbers=none]
   0x00007ffff7a87af0 <+0>:	mov    0x34f271(%rip),%rax        # 0x7ffff7dd6d68
   0x00007ffff7a87af7 <+7>:	xor    %ecx,%ecx
   0x00007ffff7a87af9 <+9>:	mov    %fs:(%rax),%r8
   0x00007ffff7a87afd <+13>:	jmpq   0x7ffff7a87b50 <*__GI_____strtol_l_internal>
\end{lstlisting}
\clearpage
\subsection{}

\clearpage
\subsection{printf}

\begin{lstlisting}[numbers=none]
(gdb) disass main
Dump of assembler code for function main:
   0x000000000040057d <+0>:     push   %rbp
   0x000000000040057e <+1>:     mov    %rsp,%rbp
   0x0000000000400581 <+4>:     sub    $0x20,%rsp
   0x0000000000400585 <+8>:     mov    %edi,-0x14(%rbp)
   0x0000000000400588 <+11>:    mov    %rsi,-0x20(%rbp)
   0x000000000040058c <+15>:    cmpl   $0x1,-0x14(%rbp)
   0x0000000000400590 <+19>:    jle    0x4005b1 <main+52>
   0x0000000000400592 <+21>:    mov    -0x20(%rbp),%rax
   0x0000000000400596 <+25>:    add    $0x8,%rax
   0x000000000040059a <+29>:    mov    (%rax),%rax
   0x000000000040059d <+32>:    mov    $0xa,%edx
   0x00000000004005a2 <+37>:    mov    $0x0,%esi
   0x00000000004005a7 <+42>:    mov    %rax,%rdi
   0x00000000004005aa <+45>:    callq  0x400430 <strtol@plt>
   0x00000000004005af <+50>:    jmp    0x4005b6 <main+57>
   0x00000000004005b1 <+52>:    mov    $0x0,%eax
   0x00000000004005b6 <+57>:    mov    %rax,-0x8(%rbp)
   0x00000000004005ba <+61>:    mov    -0x8(%rbp),%rax
   0x00000000004005be <+65>:    mov    %rax,%rdi
   0x00000000004005c1 <+68>:    callq  0x40054c <factorial>
   0x00000000004005c6 <+73>:    mov    %rax,%rdx
   0x00000000004005c9 <+76>:    mov    -0x8(%rbp),%rax
   0x00000000004005cd <+80>:    mov    %rax,%rsi
   0x00000000004005d0 <+83>:    mov    $0x40069c,%edi
   0x00000000004005d5 <+88>:    mov    $0x0,%eax
   0x00000000004005da <+93>:    callq  0x400410 <printf@plt>
   0x00000000004005df <+98>:    mov    $0x0,%eax
   0x00000000004005e4 <+103>:   leaveq
   0x00000000004005e5 <+104>:   retq
End of assembler dump.

(gdb) x/s 0x40069c
0x40069c:        "factorial(%lu) = %lu\n"
\end{lstlisting}

Wir gehen also davon aus, dass die Adresse \m{0x40069c}, die den gesuchten String enthält, die Antwort ist.
\clearpage
Der Anfang/das Ende von printf sind 0x00007ffff7a9e190/0x00007ffff7a9e231. In der Funktion findet eine Ausdehung statt und zwar auf:
\_IO\_vfprintf\_internal
Diese Funktion befindet sich dem virtuellen Adressbereich von 0x00007ffff7a93330 -   0x00007ffff7a9894d

\subsection{Adressbereiche}
Page-Größe ist 4 KiB = 4096B = 0x1000 B.\\
Offset von Page-Frame 0 ist 0x1000000.\\


Page(ADRESSE) = $\lfloor\frac{(ADRESSE)}{0x1000}\rfloor$ = ?\\\newline
Offset(ADRESSE) = (ADRESSE) mod 0x1000 = ? \\

Dann für die Adresse in der entsprechenden Page den Pageframe heraussuchen, \\ z.B. Page 0 $->$ 10342,  \newline

Physikalische Adresse = $(Pageframe) \cdot 0x1000 + (Offset) + 0x1000000$.
Berechnung der Adressen:\\

Einige Adressbereich würden keine korrekte Page-Adresse zurückliefern deswegen, haben wir auf deren Rechungen verzichtet.
\subsubsection{Adressen aus 1a)}

Wir haben in 1a) folgende Adressen gefunden: 
\begin{itemize}
\item $0x000000000040057d$; 
	\begin{itemize}
		\item Page = $\lfloor \frac{0x000000000040057d}{0x1000} \rfloor = 1024 \rightarrow PageFrame = 17363 (0x43D3)$;  
		\item Offset = $0x000000000040057d$ mod $0x1000$  = $0x57D$
		\item Phys. Adresse = $0x43D3 \cdot 0x1000 + 0x57D + 0x1000000 = 0x53D357D$
	\end{itemize}
\item $0x00000000004005e5$; 
	\begin{itemize}
		\item Page = $\lfloor \frac{0x00000000004005e5}{0x1000} \rfloor = 1024 \rightarrow PageFrame = 17363$.
		\item Offset = $0x00000000004005e5$ mod $0x1000$  = $0x5E5$
		\item Phys. Adresse = $0x43D3 \cdot 0x1000 + 0x5E5 + 0x1000000 = 0x53D35E5$
	\end{itemize}
\item $0x000000000040054c$;
	\begin{itemize}
		\item  Page = $\lfloor \frac{0x000000000040054c}{0x1000} \rfloor = 1024 \rightarrow PageFrame = 17363$.
		\item Offset = $0x000000000040054c$ mod $0x1000$  = $0x54C$
		\item Phys. Adresse = $0x43D3 \cdot 0x1000 + 0x54C + 0x1000000 = 0x53D354C$
	\end{itemize}
\item $0x000000000040057c$;
	\begin{itemize}
		\item Page =  $\lfloor \frac{0x000000000040057c}{0x1000} \rfloor = 1024 \rightarrow PageFrame = 17363$.
		\item Offset = $0x000000000040057c$ mod $0x1000$  = $0x57C$
		\item Phys. Adresse = $0x43D3 \cdot 0x1000 + 0x57C + 0x1000000= 0x53D357C$
	\end{itemize}
\end{itemize}

\subsubsection{Adressen aus 1b)}

\subsubsection{Adressen aus 1c)}
\begin{itemize}
\item $0x40069c$; 
	\begin{itemize}
		\item Page = $\lfloor \frac{0x40069c}{0x1000} \rfloor = 1024 \rightarrow PageFrame = 17363$;  
		\item Offset = $0x40069c$ mod $0x1000$  = $0x69C$
		\item Phys. Adresse = $0x43D3 \cdot 0x1000 + 0x69C + 0x1000000= 0x53D369C$
	\end{itemize}
\end{itemize}

