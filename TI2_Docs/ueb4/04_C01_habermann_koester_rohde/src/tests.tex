% !TeX root = ../04_C01_habermann_koester_rohde.tex

\subsubsection{Typische Eingabesyntax eines Unix-Shellkommandos}
Wir haben verschiedene, typische Shellkommandos zum Test eingegeben.

\begin{lstlisting}[numbers=none]
ti2sh$ ls -la
command: ls, background: nein
total 150
drwxr-xr-x 2 frohde stud    11 Nov 27 12:13 .
drwxr-xr-x 4 frohde stud     9 Nov 27 10:58 ..
-rw-r--r-- 1 frohde stud 15449 Nov 27 10:58 Aufgabe3.png
-rw-r--r-- 1 frohde stud 43496 Nov 27 10:58 Aufgabe3.vsdx
-rw-r--r-- 1 frohde stud   206 Nov 16 10:48 Makefile
-rw-r--r-- 1 frohde stud   160 Nov 16 10:48 parser.h
-rw-r--r-- 1 frohde stud  1029 Nov 16 10:48 r.l
-rw-r--r-- 1 frohde stud 26808 Nov 24 18:20 r.o
-rwxr-xr-x 1 frohde stud 67339 Nov 27 12:12 ti2sh
-rw-r--r-- 1 frohde stud  2844 Nov 27 12:12 ti2sh.cc
-rw-r--r-- 1 frohde stud 72176 Nov 27 12:12 ti2sh.o
\end{lstlisting}

\begin{lstlisting}[numbers=none]
ti2sh$ ls -la &
command: ls, background: ja
total 150
drwxr-xr-x 2 frohde stud    11 Nov 27 12:13 .
drwxr-xr-x 4 frohde stud     9 Nov 27 10:58 ..
-rw-r--r-- 1 frohde stud 15449 Nov 27 10:58 Aufgabe3.png
-rw-r--r-- 1 frohde stud 43496 Nov 27 10:58 Aufgabe3.vsdx
-rw-r--r-- 1 frohde stud   206 Nov 16 10:48 Makefile
-rw-r--r-- 1 frohde stud   160 Nov 16 10:48 parser.h
-rw-r--r-- 1 frohde stud  1029 Nov 16 10:48 r.l
-rw-r--r-- 1 frohde stud 26808 Nov 24 18:20 r.o
-rwxr-xr-x 1 frohde stud 67339 Nov 27 12:12 ti2sh
-rw-r--r-- 1 frohde stud  2844 Nov 27 12:12 ti2sh.cc
-rw-r--r-- 1 frohde stud 72176 Nov 27 12:12 ti2sh.o
ti2sh$
\end{lstlisting}

\begin{lstlisting}[numbers=none]
ti2sh$ whoami &
command: whoami, background: ja
frohde
ti2sh$
\end{lstlisting}

\begin{lstlisting}[numbers=none]
ti2sh$ whoami
command: whoami, background: nein
ti2sh$ frohde
\end{lstlisting}
\clearpage
\begin{lstlisting}[numbers=none]
ti2sh$ ping google.de
command: ping, background: nein
ti2sh$ PING google.de (172.217.20.3) 56(84) bytes of data.
64 bytes from ham02s13-in-f3.1e100.net (172.217.20.3): icmp_req=1 ttl=57 time=6.08 ms
64 bytes from ham02s13-in-f3.1e100.net (172.217.20.3): icmp_req=2 ttl=57 time=6.07 ms
64 bytes from ham02s13-in-f3.1e100.net (172.217.20.3): icmp_req=3 ttl=57 time=6.02 ms
64 bytes from ham02s13-in-f3.1e100.net (172.217.20.3): icmp_req=4 ttl=57 time=6.08 ms
64 bytes from ham02s13-in-f3.1e100.net (172.217.20.3): icmp_req=5 ttl=57 time=6.04 ms
64 bytes from ham02s13-in-f3.1e100.net (172.217.20.3): icmp_req=6 ttl=57 time=6.07 ms
64 bytes from ham02s13-in-f3.1e100.net (172.217.20.3): icmp_req=7 ttl=57 time=6.07 ms
64 bytes from ham02s13-in-f3.1e100.net (172.217.20.3): icmp_req=8 ttl=57 time=6.07 ms
64 bytes from ham02s13-in-f3.1e100.net (172.217.20.3): icmp_req=9 ttl=57 time=6.08 ms
^C
--- google.de ping statistics ---
9 packets transmitted, 9 received, 0% packet loss, time 8010ms
rtt min/avg/max/mdev = 6.023/6.068/6.084/0.105 ms
\end{lstlisting}



\subsubsection{``Mit dem Kopf über die Tastatur''}
Die Eingabe beliebiger, schwachsinniger Kommandos erzeugt keine Rückgabe, bei aktivierten Debug-Mode sieht man, dass die Ausführung fehlschlägt.
\begin{lstlisting}[numbers=none]
ti2sh$ blafasel
command: blafasel, background: nein
ti2sh$ ti2sh$ bla /fasel
command: bla, background: nein
ti2sh$ ti2sh$
\end{lstlisting}

\subsubsection{Ein-/Ausgabeumlenkung}
...war nicht zu implementieren:
\begin{lstlisting}[numbers=none]
ti2sh$ date > bullshit
ti2sh: Ein-/Ausgabe-Umlenkung nicht implementiert!
ti2sh$ bullshit | date
ti2sh: Ein-/Ausgabe-Umlenkung nicht implementiert!
\end{lstlisting}

\clearpage

\subsubsection{Absoluter Pfad}
Unser \C{foldermode} funktioniert... :-)
\begin{lstlisting}[numbers=none]
ti2sh$ /bin/ls -la
command: /bin/ls, background: nein
ti2sh$ total 150
drwxr-xr-x 2 frohde stud    11 Nov 27 12:13 .
drwxr-xr-x 4 frohde stud     9 Nov 27 10:58 ..
-rw-r--r-- 1 frohde stud 15449 Nov 27 10:58 Aufgabe3.png
-rw-r--r-- 1 frohde stud 43496 Nov 27 10:58 Aufgabe3.vsdx
-rw-r--r-- 1 frohde stud   206 Nov 16 10:48 Makefile
-rw-r--r-- 1 frohde stud   160 Nov 16 10:48 parser.h
-rw-r--r-- 1 frohde stud  1029 Nov 16 10:48 r.l
-rw-r--r-- 1 frohde stud 26808 Nov 24 18:20 r.o
-rwxr-xr-x 1 frohde stud 67339 Nov 27 12:12 ti2sh
-rw-r--r-- 1 frohde stud  2844 Nov 27 12:12 ti2sh.cc
-rw-r--r-- 1 frohde stud 72176 Nov 27 12:12 ti2sh.o

ti2sh$
\end{lstlisting}

Da wir aber nur das 1. Element der Eingabe auf Slashes überprüfen, sollten die Parameter eines Programmes diese enthalten dürfen:

\begin{lstlisting}[numbers=none]
ti2sh$ ls /var/tmp
command: ls, background: nein
auto_logoff.sh  elvis13.ses  elvis18.ses  elvis22.ses  elvis27.ses  elvis31.ses  elvis6.ses        libcoap
elvis1.ses      elvis14.ses  elvis19.ses  elvis23.ses  elvis28.ses  elvis32.ses  elvis7.ses
elvis10.ses     elvis15.ses  elvis2.ses   elvis24.ses  elvis29.ses  elvis33.ses  elvis8.ses
elvis11.ses     elvis16.ses  elvis20.ses  elvis25.ses  elvis3.ses   elvis4.ses   elvis9.ses
elvis12.ses     elvis17.ses  elvis21.ses  elvis26.ses  elvis30.ses  elvis5.ses   kdecache-hen_mai
\end{lstlisting}
\subsubsection{Produktivität}
Es ist problemlos möglich, einen \C{nano} zu starten und irgendeine Datei zu schreiben. Jedoch ist uns aufgefallen, dass man \m{Ctrl-X} zweimal drücken muss, um \C{nano} zu beenden.
\begin{lstlisting}[numbers=none]
ti2sh$ nano ti2rockt.md
command: nano, background: nein
                                            [GNU nano 2.2.6         File:ti2rockt.md]
^X
ti2sh$
\end{lstlisting}
