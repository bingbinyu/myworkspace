\documentclass{ti2}

\usepackage{listings}

% Dateikodierung ist utf8
\usepackage[utf8]{inputenc}   

\begin{document}

%damit der Code nicht abgeschnitten wird, falls er zu lang ist.
\lstset{linewidth=\linewidth,breaklines=true}
% Nr, Abgabedatum, Gruppenleiter, Gruppenname, Name1...Name4
\Abgabeblatt{2}{14.11.2016}{Marc Hildebrandt}{C08}%
                {Timo Jasper (Inf, 3.FS.)}{Thomas Tannous (Inf, 3.FS.)}%
                {Oliver Hilbrecht (Inf, 3.FS.)}{Moritz Gerken (Inf, 3.FS.)}%

\section{Aufgabe 1}
Hier haben wir mit den Befehlen cat, tail und cut die nötigen Stellen
aus dem Tag rausgeschnitten, die wir benötigten. 
Dabei ist der String leer, falls nichts an der Stelle keine Angabe war.
So haben wir wie erwartet die Angabe durch ein \_ ersetzt.
Für die Titelnummer haben wir den Befehl dd benutzt um die nötigen
Bytes zu bekommen. Mit \lstinline{hexdump} -e .. kriegen wir dann die Tracknummer aus dem Binärcode in eine Dezimal konvertiert.
\lstinputlisting{src/aufgabe1/mp3-rename.sh}
\subsection{Tests}
Unsere Tests prüfen auf die wichtigen Randfälle. \\
\lstinputlisting{src/aufgabe1/testarea.sh}
\section{Aufgabe 2}%Was hier zu tun ist seht ihr in den Folien zu Vorlesung_4

\subsection{main}

\subsubsection{a (Text/Data-Relocation-Table, Symboltabelle und Stringtabelle für das a.out-Format)}

Textsegment:
\begin{tabular}{|c|l l l|}
	\hline
	0 & & & pushq	\%rbp\\
	\hline
	4 & & & movq	\%rsp, \%rbp\\
	\hline
	8 & & & subq	\$32, \%rsp\\
	\hline
	12 & & & movl	\%edi, -20(\%rbp)\\
	\hline
	16 & & & movq	\%rsi, -32(\%rbp)\\
	\hline
	20 & & & cmpl	\$1, -20(\%rbp)\\
	\hline
	24 & & & jle	+72\\
	\hline
	28 & & & movq	-32(\%rbp), \%rax\\
	\hline
	32 & & & addq	\$8, \%rax\\
	\hline
	36 & & & movq	(\%rax), \%rax\\
	\hline
	40 & & & movl	\$10, \%edx\\
	\hline
	44 & & & movl	\$0, \%esi\\
	\hline
	48 & & & movq	\%rax, \%rdi \\
	\hline
	52 & & & call	0 \\
	\hline
	56 & & & movq	\%rax, -16(\%rbp)\\
	\hline
	60 & & & movq	\$0, -8(\%rbp) \\
	\hline
	64 & & & movq	-8(\%rbp), \%rax\\
	\hline
	68 & & & cmpq	-16(\%rbp), \%rax\\
	\hline
	72 & & & ja	+32\\
	\hline
	76 & & & movq	-8(\%rbp), \%rax\\
	\hline
	80 & & & movq	\%rax, \%rdi\\
	\hline
	84 & & & call	0\\
	\hline
	88 & & & movq	\%rax, \%rdi\\
	\hline
	92 & & & call	show\_\_Fm\\
	\hline
	96 & & & addq	\$1, -8(\%rbp)\\
	\hline
	100 & & & jmp	-36\\
	\hline
	104 & & & movq	0(\%rip), \%rax\\
	\hline
	108 & & & movq	\%rax, \%rsi\\
	\hline
	112 & & & leaq	cout(\%rip), \%rdi\\
	\hline
	116 & & & call	0\\
	\hline
	120 & & & movl	\$0, \%eax\\
	\hline
	124 & & & ret \\
	\hline
\end{tabular}\\ \\ \\ \\ \\

Datensegment, Relocation Tabelle, SymbolTabelle, String Tabelle :\\
\begin{tabular}{|c| l l l|}	
	\hline
	& \multicolumn{3}{|c|}{(Datensegment ist leer)} \\
	\hline
	& 52 & 1 & \\
	\hline
	& 84 & 2 & \\
	\hline
	& 92 & 3 & \\
	\hline
	& 116 & 4 & \\
	\hline
	0 & 4 & Text & 0 \\
	\hline
	1 & 9 & UNDEFINED &  \\
	\hline
	2 & 23 & UNDEFINED &  \\
	\hline
	3 & 31 & UNDEFINED &  \\
	\hline
	4 & 40 & UNDEFINED &  \\
	\hline
	0 & \multicolumn{3}{|c|}{Länge (4 Bytes)} \\
	\hline
	4 & main & &  \\
	\hline
	8 & $\backslash0$str & & \\
	\hline
	12 & tol\_ & &  \\
	\hline
	16 & \_Fxc & & \\
	\hline
	20 & ci$\backslash0$f & &  \\
	\hline
	24 & ib\_\_ & & \\
	\hline
	28 & Fm$\backslash0$s & &  \\
	\hline
	32 & how\_ & & \\
	\hline
	36 & \_Fm$\backslash0$ & &  \\
	\hline
	40 & \_\_ls & & \\
	\hline
	44 & \_\_7o & &  \\
	\hline
	48 & stre & & \\
	\hline
	52 & amm$\backslash0$ & & \\
	\hline
\end{tabular}


\subsection{show}

\subsubsection{a (Text/Data-Relocation-Table, Symboltabelle und Stringtabelle für das a.out-Format)}



\begin{tabular}{|c|l l l|}
	\hline
	0 & & & pushq	\%rbp\\
	\hline
	4 & & & movq	\%rsp, \%rbp\\
	\hline
	8 & & & subq	\$16, \%rsp\\
	\hline
	12 & & & movq	\%rdi, -8(\%rbp)\\
	\hline
	16 & & & movq	-8(\%rbp), \%rax\\
	\hline
	20 & & & movq	\%rax, \%rsi\\
	\hline
	24 & & & leaq	cout(\%rip), \%rdi\\
	\hline
	28 & & & call	0\\
	\hline
	32 & & & movl	\$32, \%esi\\
	\hline
	36 & & & movq	\%rax, \%rdi\\
	\hline
	40 & & & movq	\%rax, \%rdi\\
	\hline
	44 & & & call	0\\
	\hline
	48 & & & ret\\
	\hline
	& \multicolumn{3}{|c|}{(Datensegment ist leer)} \\
	\hline
	& 28 & 1 & \\
	\hline
	& 44 & 2 & \\
	\hline
	0 & 4 & Text & 0 \\
	\hline
	1 & 13 & UNDEFINED &  \\
	\hline
	2 & 29 & UNDEFINED &  \\
	\hline
	0 & \multicolumn{3}{|c|}{Länge (4 Bytes)} \\
	\hline
	4 &  & show &  \\
	\hline
	8 & & \_\_Fm & \\
	\hline
	12 & & \textbackslash0\_\_l & \\
	\hline
	16 & & s\_\_7 & \\
	\hline
	20 & & ostr & \\
	\hline
	24 & & eamm & \\
	\hline
	28 & & \textbackslash0\_\_l & \\
	\hline
	32 & & s\_\_7 & \\
	\hline
	36 & & ostr & \\
	\hline
	40 & & eamc & \\
	\hline
	44 & & \textbackslash & \\
	\hline
\end{tabular}	



\subsection{fib}

\subsubsection{a (Text/Data-Relocation-Table, Symboltabelle und Stringtabelle für das a.out-Format)}


\begin{tabular}{|c|l l l|}
	\hline
	0 & & & pushq	\%rbp\\
	\hline
	4 & & & movq	\%rsp, \%rbp\\
	\hline
	8 & & & pushq	\%rbx\\
	\hline
	12 & & & subq	\$40, \%rsp\\
	\hline
	16 & & & movq	\%rdi, -40(\%rbp)\\
	\hline
	20 & & & movq	\$1, -24(\%rbp)\\
	\hline
	24 & & & cmpq	\$1, -40(\%rbp)\\
	\hline
	28 & & & jbe	+48\\
	\hline
	32 & & & movq	-40(\%rbp), \%rax\\
	\hline
	36 & & & subq	\$2, \%rax\\
	\hline
	40 & & & movq	\%rax, \%rdi\\
	\hline
	44 & & & call	-44\\
	\hline
	48 & & & movq	\%rax, \%rbx\\
	\hline
	52 & & & movq	-40(\%rbp), \%rax\\
	\hline
	56 & & & subq	\$1, \%rax\\
	\hline
	60 & & & movq	\%rax, \%rdi\\
	\hline
	64 & & & call	-64\\
	\hline
	68 & & & addq	\%rbx, \%rax\\
	\hline
	72 & & & movq	\%rax, -24(\%rbp) \\
	\hline
	76 & & & movq	-24(\%rbp), \%rax\\
	\hline
	80 & & & addq	\$40, \%rsp\\
	\hline
	84 & & & popq	\%rbx\\
	\hline
	88 & & & popq	\%rbp\\
	\hline
	92 & & & ret\\
	\hline
	& \multicolumn{3}{|c|}{(Datensegment ist leer)} \\
	\hline
	& \multicolumn{3}{|c|}{(ext/Data Rel. Tab ist leer)} \\
	\hline
	0 & 4 & Text & 0 \\
	\hline
	0 & \multicolumn{3}{|c|}{Länge (4 Bytes)} \\
	\hline
	4 &  & fib\_ &  \\
	\hline
	8 & & \_Fm\textbackslash0 & \\
	\hline
\end{tabular}

\subsection{b}
\begin{tabular}{|c|l l l|}
	\hline
	0 & & & pushq	\%rbp\\
	\hline
	4 & & & movq	\%rsp, \%rbp\\
	\hline
	8 & & & subq	\$16, \%rsp\\
	\hline
	12 & & & movq	\%rdi, -8(\%rbp)\\
	\hline
	16 & & & movq	-8(\%rbp), \%rax\\
	\hline
	20 & & & movq	\%rax, \%rsi\\
	\hline
	24 & & & leaq	cout(\%rip), \%rdi\\
	\hline
	28 & & & call	0\\
	\hline
	32 & & & movl	\$32, \%esi\\
	\hline
	36 & & & movq	\%rax, \%rdi\\
	\hline
	40 & & & movq	\%rax, \%rdi\\
	\hline
	44 & & & call	0\\
	\hline
	48 & & & ret\\
	\hline
	& \multicolumn{3}{|c|}{(Datensegment ist leer)} \\
	\hline
	& 28 & 1 & \\
	\hline
	& 44 & 2 & \\
	\hline
	0 & 4 & Text & 0 \\
	\hline
	1 & 13 & UNDEFINED &  \\
	\hline
	2 & 29 & UNDEFINED &  \\
	\hline
	0 & \multicolumn{3}{|c|}{Länge (4 Bytes)} \\
	\hline
	4 &  & show &  \\
	\hline
	8 & & \_\_Fm & \\
	\hline
	12 & & \textbackslash0\_\_l & \\
	\hline
	16 & & s\_\_7 & \\
	\hline
	20 & & ostr & \\
	\hline
	24 & & eamm & \\
	\hline
	28 & & \textbackslash0\_\_l & \\
	\hline
	32 & & s\_\_7 & \\
	\hline
	36 & & ostr & \\
	\hline
	40 & & eamc & \\
	\hline
	44 & & \textbackslash & \\
	\hline
\end{tabular}	

\subsection{c}

\section{Aufgabe 3}

Da man den Wert der Variable my\_password nicht direkt auslesen kann, weil sie "optimized out" ist
(was so viel heißt wie der Code wurde optimiert und der Wert ist nicht einfach zugänglich), 
mussten wir einen anderen Weg wählen. 
Zuerst haben wir nach der Funktion geschaut die unser Passwort auf Gleichheit prüft.
Dies war die check\_password(char const*) Funktion. Anschließend haben wir sie disassembled.
Dort haben dann nach dem vergleich geguckt und die Addresse des Befehls notiert.
An der stelle haben wir dann ein Breakpoint gesetzt und das Programm laufen lassen.
Als gebreakt wurde, schauten wir was im Register \$rsi nach einem String mit dem Befehl 
x/s \$rsi. Dort befand sich dann zum Glück das Passwort "TeI2-ist\_einfach". 

GDB weiß so viel weil:
Der Maschinencode kann zurück in assembler übersetzt werden.
Die binary Datei selbst hat einen Teil der sich das Datensegment nennt, wo alle initialisierten Daten vorhanden sind. 
So kann gdb den Maschinencode verstehen, einen breakpoint setzen und Register des Prozessors auslesen. 

\section{Weitere Aufgaben}



\end{document}
